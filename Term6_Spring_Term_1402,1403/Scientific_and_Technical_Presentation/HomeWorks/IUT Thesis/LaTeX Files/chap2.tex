% Chapter 2
\chapter{تحلیل شبکه‌های پیچیده}
\section{توزیع درجه\footnote{Degree distribution}}
بررسی توزیع درجه گره‌ها در شبکه بیت‌کوین بسیار حائز اهمیت است. درجه یک گره که با \(k\) نشان داده می‌شود، تعداد یال‌های مجاور آن گره را مشخص می‌کند. در شبکه تراکنش‌های بیت‌کوین، درجه \(k\) برای هر آدرس بیت‌کوین با مجموع تعداد تراکنش‌ها محاسبه می‌شود. تعداد تراکنش‌های ورودی (دریافت بیت‌کوین) به عنوان درجه ورودی و تعداد تراکنش‌های خروجی (پرداخت بیت‌کوین) به عنوان درجه خروجی شناخته می‌شود. توزیع درجه که با \(P(k)\) نشان داده می‌شود، احتمال این است که یک گره انتخابی به صورت تصادفی دارای درجه برابر با \(k\) باشد. اگر درجه \(k\) از قانون توان تبعیت کند، آنگاه \(P(k) \propto k^{-\alpha}\) است، که \(\alpha\) پارامتر مقیاس توزیع قانون توان است.

نتایج نشان می‌دهند که تمامی توزیع‌های درجه از قانون توان با دنباله‌های سنگین تبعیت می‌کنند. این نشان می‌دهد که شبکه بیت‌کوین یک شبکه مقیاس آزاد است که در آن تنها تعداد کمی از گره‌ها دارای تعداد زیادی اتصالات هستند در حالی که بیشتر گره‌ها دارای درجه‌های کم و اتصالات کمتری هستند. این یافته‌ها با نتایج تحقیقاتی که با داده‌های واقعی شبکه به دست آمده‌اند، سازگار است. 


\section{ضریب خوشه‌بندی و طول کوتاه‌ترین مسیر\footnote{Clustering coefficient and the shortest-path length}}

ضریب خوشه‌بندی و طول کوتاه‌ترین مسیر می‌توانند شبکه را از دیدگاه هندسی مورد ارزیابی قرار دهند. ضریب خوشه‌بندی متوسط شبکه با \(C\) نشان داده می‌شود که به صورت زیر تعریف می‌شود:
\[
C = \frac{1}{N} \sum_{i \in V(G)} \frac{\Delta_i}{k_i (k_i - 1) / 2},
\]
که در آن \(N\) تعداد گره‌ها، \(\Delta_i\) تعداد مثلث‌های کامل و \(k_i\) درجه گره \(i\) را نشان می‌دهند.

از سوی دیگر، طول متوسط کوتاه‌ترین مسیر با \(L\) نشان داده می‌شود که به صورت زیر تعریف می‌شود:
\[
L = \sum_{i,j \in V(G)}  \frac{l(i,j)}{N(N-1)},
\]
که در آن \(V(G)\) مجموعه گره‌های گراف \(G\) و \(l(i,j)\) طول کوتاه‌ترین مسیر بین \(i\) و \(j\) است. برای گراف تراکنش بیت‌کوین، ضریب خوشه‌بندی \(C_{Bitcoin} = 0.0071\) و طول کوتاه‌ترین مسیر \(L_{Bitcoin} = 3.833\) می‌باشد که نشان‌دهنده تعداد زیاد تراکنش‌های غیرمستقیم است.

\subsection{اثر جهان کوچک در شبکه بیت‌کوین\footnote{Small-world effect}}

اثر جهان کوچک یک ویژگی در شبکه‌های پیچیده است که نشان می‌دهد چگونه هر دو گره در یک شبکه بزرگ می‌توانند با تعداد کمی یال به یکدیگر متصل شوند. دو ویژگی اصلی شبکه‌های جهان کوچک عبارتند از:

\begin{itemize}
    \item \textbf{ضریب خوشه‌بندی بالا:} ضریب خوشه‌بندی نشان می‌دهد که چقدر احتمال دارد که دو گره همسایه یک گره دیگر نیز با هم متصل باشند. ضریب خوشه‌بندی بالا نشان‌دهنده وجود خوشه‌های محکم از گره‌ها است.
    \item \textbf{میانگین طول کوتاه‌ترین مسیر کم:} این ویژگی نشان می‌دهد که به طور میانگین، چند یال باید طی شود تا از یک گره به هر گره دیگر در شبکه رسید. در شبکه‌های جهان کوچک، این میانگین طول کوتاه است.
\end{itemize}

در شبکه بیت‌کوین، ضریب خوشه‌بندی و طول کوتاه‌ترین مسیر به صورت زیر محاسبه شده است:

\[
C_{\text{Bitcoin}} = 0.0071 \quad \text{و} \quad L_{\text{Bitcoin}} = 3.833
\]

این مقادیر نشان‌دهنده این است که شبکه بیت‌کوین دارای خوشه‌های محکم از گره‌ها و همچنین مسیرهای کوتاه بین گره‌ها است. بنابراین در شبکه پیچیده بیت‌کوین شاهد اثر جهان کوچک می‌باشیم. این اثر به این معناست که توکن‌های بیت‌کوین\footnote{Bitcoin tokens} می‌توانند در چند مرحله به اکثر گره‌ها منتقل شوند.

~\vfill
\section{مولفه‌های متصل\footnote{Connected component}}
با توجه به اینکه شبکه بلاکچین بیت‌کوین یک شبکه جهان کوچک است، تحلیل اتصال‌پذیری آن اهمیت زیادی دارد. در یک شبکه پیچیده، اگر هر جفت گره در یک زیرگراف حداقل یک مسیر متصل داشته باشد، آن زیرگراف را یک مولفه متصل می‌نامیم. در شبکه‌های جهت‌دار، مولفه‌های قویاً متصل\footnote{Strongly Connected Component(SCC)} به زیرگراف‌هایی اشاره دارد که هر جفت گره \text{($i , j$)} دارای مسیری جهت‌دار از $i$ به $j$ و از $j$ به $i$ به‌طور همزمان هستند. به‌طور مشابه، مولفه‌های ضعیفاً متصل\footnote{Weakly Connected Component(WCC)} به مولفه‌های متصل بدون جهت اشاره دارد.

نتایج حاصل از این تحلیل نشان می‌دهند که گراف شبکه پیچیده بیت‌کوین یک گراف نسبتا متصل است. همچنین احتمالاً گره‌های رابط، تعداد بسیاری از گره‌های جدا شده و منفرد را به شبکه متصل می‌کنند. در واقعیت، چنین گره‌های رابطی ممکن است صرافی‌ها، موسسات تجاری یا سازمان‌های مالی باشند. همچنین می‌توان نتیجه گرفت که بسیاری از تراکنش‌ها در این گراف تنها یک‌طرفه هستند. به عبارت دیگر، اکثر گره‌ها به‌طور مکرر تراکنش‌های دوطرفه (ورودی و خروجی) انجام نمی‌دهند و فقط بیت‌کوین پرداخت می‌کنند یا دریافت می‌کنند.



~\section{مرکزیّت\footnote{Centrality}}
تحلیل مولفه‌های متصل باعث شد به وجود گره‌های رابط پی ببریم. برای تأیید این فرضیه، مرکزیّت شبکه را تحلیل می‌کنیم.

~\subsection{مرکزیّت شبکه}
مرکزیّت شبکه مفهومی است که برای اندازه‌گیری اهمیت نسبی گره‌ها در یک شبکه استفاده می‌شود. این مفهوم به ما کمک می‌کند تا بفهمیم کدام گره‌ها نقش کلیدی‌تری در ساختار شبکه ایفا می‌کنند. در ادامه به بررسی چند نوع مرکزیّت در شبکه‌های پیچیده می‌پردازیم.

\subsubsection{مرکزیّت نزدیکی\footnote{Closeness centrality }}
مرکزیّت نزدیکی معیاری است که نشان می‌دهد یک گره چقدر به سایر گره‌های شبکه نزدیک است. این معیار بر اساس طول کوتاه‌ترین مسیرها از یک گره به سایر گره‌ها محاسبه می‌شود. فرمول مرکزیّت نزدیکی یک گره $i$ به صورت زیر است:
\[
O(i) = \frac{n-1}{\sum_{j=1}^{n-1} d(i, j)}
\]
که در آن $n$ تعداد گره‌های قابل دسترس گره $i$ و $d(j, i)$ فاصله کوتاه‌ترین مسیر بین گره $j$ و گره $i$ است. این معیار نشان می‌دهد که یک گره چقدر سریع می‌تواند به سایر گره‌ها دسترسی پیدا کند.

\subsubsection{مرکزیّت بینابینی\footnote{Betweenness centrality}}
مرکزیّت بینابینی نشان می‌دهد که یک گره چقدر در مسیرهای کوتاه بین سایر گره‌ها قرار دارد. این معیار نشان می‌دهد که یک گره چقدر در انتقال اطلاعات بین سایر گره‌ها نقش دارد. فرمول مرکزیّت بینابینی یک گره $i$ به صورت زیر است:
\[
B(i) = \sum_{u, v \in V} \frac{\sigma(u, v | i)}{\sigma(u, v)}
\]
که در آن $\sigma(u, v)$ تعداد کل مسیرهای کوتاه بین گره‌های $u$ و $v$ و $\sigma(u, v | i)$ تعداد مسیرهایی است که از گره $i$ عبور می‌کنند. این معیار نشان می‌دهد که یک گره چقدر در اتصال سایر گره‌ها به هم نقش دارد.
\\
\\
مطابق با یافته‌های این مقاله، مرکزیّت بینابینی با افزایش درجه گره افزایش می‌یابد. اگر تعداد زیادی از گره‌ها دارای مقادیر بالای بینابینی باشند، تعداد زیادی از گره‌های رابط در گراف ظاهر می‌شوند که باعث شکنندگی گراف می‌شود. این نتایج نشان می‌دهد که تعداد زیادی از گره‌های رابط در شبکه بیت‌کوین وجود ندارد و این شبکه در مقابل حذف گره‌ها مقاوم است. بیشتر گره‌ها دارای مقادیر نسبتاً کم نزدیکی و بینابینی هستند که نشان می‌دهد تعداد کمی گره‌های مرکزی وجود دارند. بنابراین، ما یک گراف چندمرکزی مشاهده می‌کنیم که در آن برخی از گره‌های مرکزی مستقیماً با تعداد زیادی از گره‌ها بدون واسطه متصل هستند. دلیل چندمرکزی و مقاومت عالی را می‌توان به توزیع ناهمگن گره‌ها نسبت داد که در بخش‌های بعدی مورد بررسی قرار می‌گیرد.


\section{عدم تناسب\footnote{Disassortativity}}
در این تحلیل تمایلات اتصالات شبکه بیت‌کوین مورد بررسی قرار گرفته است. محققان از ضریب همبستگی پیرسون\footnote{Pearson correlation coefficient} با نماد $\rho$ برای مشخص کردن تناسب شبکه استفاده کرده‌اند.

% این ضریب به‌شکل زیر محاسبه می‌شود:
%\begin{equation}
%\rho = \frac{\sum_{e_{ij} \in E(G)} (k_i k_j - \frac{1}{2}(k_i + k_j))^2}{\sum_{e_{ij} \in E(G)} \frac{1}{2}(k_i^2 + k_j^2) - \left( \sum_{e_{ij} \in E(G)} \frac{1}{2}(k_i + k_j) \right)^2}
%\end{equation}

نتیجه به‌دست آمده برای ضریب همبستگی پیرسون $\rho = -0.023$ بوده که نشان‌دهنده این است که شبکه دارای خاصیت عدم تناسب است. مطالعات قبلی بر روی ساختار شبکه بیت‌کوین نیز این موضوع را تایید می‌کنند.

درنهایت می‌توان نتیجه گرفت که گره‌های با درجه بالا ترجیح می‌دهند به گره‌های با درجه کمتر متصل شوند، درحالی‌که گره‌های با درجه پایین نیز ترجیح دارند به گره‌های با درجه بالاتر متصل شوند.
به عنوان مثال، گره‌های تازه وارد ترجیح می‌دهند با گره‌های درجه بالا (که احتمالا صرافی ها و غیره می‌باشند) متصل شوند.


\section{ضریب باشگاه ثروت‌مندان\footnote{Rich-club coefficient}}
پدیده باشگاه ثروتمندان در شبکه‌های پیچیده به پدیده‌ای اطلاق می‌شود که ارتباطات قوی بین گره‌های با درجه‌های بالا را نشان می‌دهد، به‌طوری‌که این گره‌ها به صورت زیرگروه‌هایی به نام "باشگاه‌های ثروتمندان" با هم ارتباط‌ برقرار می‌کنند.

ضریب باشگاه ثروتمندان $\phi(k)$ به عنوان یک معیار برای ارزیابی این پدیده تعریف می‌شود. این ضریب نسبت تعداد لینک‌های واقعی بین گره‌هایی را که درجه آن‌ها بیشتر از $k$ است به تعداد لینک‌های حداکثر ممکن بین این گره‌ها در شبکه با $N$ گره نشان می‌دهد:
\[
\phi(k) = \frac{2E>k}{N>k (N>k - 1)},
\]
که در آن $N>k$ تعداد کل گره‌هایی است که درجه آن‌ها بیشتر از $k$ است و $E>k$ تعداد یال‌های بین گره‌های $N>k$ است. $N>k (N>k - 1)/2$ حداکثر یال‌های ممکن بین تمام گره‌های $N>k$ است.

برای ارزیابی دقیق‌تر، از ضریب باشگاه ثروت‌مند نرمال‌شده $\phi_{\text{norm}}(k)$ استفاده می‌کنیم که به صورت زیر تعریف می‌شود:
\[
\phi_{\text{norm}}(k) = \frac{\phi(k)}{\phi_{\text{rand}}(k)},
\]
که در آن $\phi_{\text{rand}}(k)$ ضریب باشگاه ثروت‌مندان با توزیع درجه مشابه است. نتایج نشان می‌دهد که در اکثر ارزش‌های $k$، چیدمان باشگاه ثروت‌مندان وجود ندارد.

در کل، شبکه بیت‌کوین پدیده عدم وجود باشگاه ثروت‌مند را از خود نشان می‌دهد، که نشان‌دهنده آن است که گره‌های مرکزی با درجه بالا در این شبکه تمایل به اتصال با یکدیگر ندارند و در زیرگراف‌های متصل مختلف پخش می‌شوند. این اثر می‌تواند توسط واقعیت توضیح داده شود که گره‌های مرکزی احتمالاً تبادلات متداولی را انجام می‌دهند. در نتیجه، گره‌های ثروتمند در شبکه بیت‌کوین به طور مستقیم با یکدیگر ارتباط ندارند.

%%%%%%%%%%%%%%%%%%%%%%%%%%%%%
\chapter{جمع‌بندی}
\section{نتایج}

در این مقاله، ما به تحلیل شبکه‌های پیچیده بر روی شبکه تراکنش‌های بیت‌کوین پرداختیم. به طور خاص، یک روش نمونه‌برداری جدید با نام \textit{پیاده‌روی تصادفی با بازگشت به عقب} طراحی کرده‌ایم و از طریق تحلیل گراف‌های نمونه‌برداری شده، مشاهدات مهمی به‌دست آورده‌ایم.

ابتدا، توزیع درجه شبکه تراکنش‌های بیت‌کوین به توزیع قانون توان با دنباله سنگین تطابق دارد که به یک شبکه مقیاس آزاد نزدیک است. دوم، با تحلیل ضریب خوشه‌بندی میانگین، طول کوتاه‌ترین مسیر و اندازه‌گیری شبکه جهان کوچک، اطمینان حاصل کردیم که شبکه تراکنش‌های بیت‌کوین یک شبکه جهان کوچک است. سوم، از طریق تحلیل اجزای متصل، دریافتیم که بیشتر تراکنش‌ها به صورت معاملات یک‌طرفه هستند. علاوه بر این، مشاهده کردیم که شبکه بیت‌کوین یک شبکه چندمرکزی مقاوم در برابر حذف گره‌ها است. 

پس از آن، با توجه به ناهماهنگی شبکه بیت‌کوین، دریافتیم که گره‌های با درجه پایین تمایل به اتصال به گره‌های با درجات بالاتر دارند. در نهایت، شبکه تراکنش‌های بیت‌کوین فعلی پدیده باشگاه ثروتمندان را نشان نمی‌دهد. این یافته‌ها می‌تواند به ما در درک بهتر رفتارهای ساختاری شبکه‌های بلاکچین کمک کند. علاوه بر این، روش‌های تحلیل و الگوریتم‌های نمونه‌برداری نیز می‌توانند در فهم شبکه‌هایی با خصوصیات مشابه، مانند شبکه‌های مقیاس آزاد، شبکه‌های جهان کوچک و شبکه‌های ناهماهنگ، بینش‌های ارزشمندی ارائه ‌دهند.
