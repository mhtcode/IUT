%*************************************************
% In this file the abstract is typeset.
% Make changes accordingly.
%*************************************************

\addcontentsline{toc}{section}{چکیده}
\newgeometry{left=2.5cm,right=3cm,top=3cm,bottom=2.5cm,includehead=false,headsep=1cm,footnotesep=.5cm}
\setcounter{page}{1}
\pagenumbering{arabic}						% شماره صفحات با عدد
\thispagestyle{empty}

~\vfill

~\subsection*{چکیده}
\begin{large}
\baselineskip=0.7cm
تحلیل شبکه‌های پیچیده\LTRfootnote { Complex Network } به عنوان یکی از شاخه‌های نوین علم شبکه‌ها، به بررسی ساختار و پویایی شبکه‌های پیچیده می‌پردازد. شبکه تراکنش بیت‌کوین\LTRfootnote { Bitcoin
Transaction Network }  به عنوان یکی از بزرگترین و مهم‌ترین شبکه‌های مالی دیجیتال، نمونه‌ای بارز از یک شبکه پیچیده است که تحلیل آن می‌تواند دیدگاه‌های ارزشمندی در مورد رفتار کاربران و ساختار کلان این شبکه فراهم کند. در این پژوهش، به بررسی شبکه تراکنش بیت‌کوین از جایگاه تحلیل شبکه‌های پیچیده پرداخته شده است.

در این مقاله، شبکه‌ی پیچیده‌ی تراکنش‌های بیت‌کوین را مورد بررسی و تحلیل قرار می‌دهیم. به‌طور خاص، یک روش نمونه‌گیری جدید به نام پیمایش تصادفی با بازگشت\LTRfootnote { Random Walk
With Flying-Back (RWFB)  } معرفی می‌شود تا نمونه‌گیری داده‌ها به‌طور موثرتری انجام شود. سپس، تحلیل جامعی از شبکه بلاکچین\LTRfootnote { Blockchain } بیت‌کوین از نظر توزیع درجه، ضریب خوشه‌بندی، طول کوتاه‌ترین مسیر، مولفه‌های متصل، مرکزیت، خودهمبستگی، و ضریب باشگاه ثروتمندان انجام می‌دهیم. پس از تحلیل، شاهد چندین نتیجه‌گیری جالب و حیرت انگیز مانند پدیده دنیای کوچک، وضعیت چند مرکزی، اتصال ترجیحی، و عدم تاثیر باشگاه ثروتمندان در شبکه فعلی خواهیم بود.


\vspace{0.5 cm}

\noindent\textbf{واژه‌های کلیدی:}
1- بیت‌کوین، 2- بلاکچین، 3- شبکه پیچیده، 4-تحلیل شبکه.
\end{large}

\vspace*{-10cm} % Add negative space to bring the content closer to the footnote